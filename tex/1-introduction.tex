%!TEX root = ../Thesis.tex
\chapter{Introduction}\label{cha:introduction}
%
Here's a citation: \cite{Wacker2016}. Two pictures are shown, Figures \ref{fig:morten} and \ref{fig:eps}. There's also a table: Table \ref{tab:a-table}.

\section{A Section}\label{sec:a-section}

\subsection{A Subsection}\label{ssec:a-subsection}

\subsubsection{A Subsubsection}\label{sssec:a-subsubsection}

\lipsum[2]

\begin{figure}[htbp]
  \centering
  \includegraphics[width=.3\textwidth]{img/morten}
  \caption{Some dude}
  \label{fig:morten}
\end{figure}

\begin{figure}[htbp]
  \centering
  \includegraphics[width=.7\textwidth]{img/time-constant}
  \caption{An eps image}
  \label{fig:eps}
\end{figure}

\begin{table}[htbp]
  \centering
  \caption{A table}
  \begin{tabular}{cc}
    \toprule
    Something & Something else \\
    \midrule
    A & a \\
    B & b \\
    \bottomrule
  \end{tabular}
  \label{tab:a-table}
\end{table}

\begin{equation*}
  \begin{aligned}
      & \underset{\*x, \*u}{\text{min}}
      & & \int_{t_k}^{t_{k+T}} ( \*{e}(\*{x})^{\top} \*{Q}_p \*{e}(\*{x}) +
      \dot{\*{e}}(\*{x})^{\top} \*{Q}_d \dot{\*{e}}(\*{x}) +
      \ddot{\*{q}}^{\top} \*{R}_{\ddot{\*{q}}} \ddot{\*{q}} +
      \*{u}^{\top} \*{R}_{\*{u}} \*{u} ) \,dt\\
      & \text{s.t.}
      & & \dot{\*x} = \*f(\*x,\*u), \\
      &&& \bar{\*x}_{lb} \leq \*x \leq \bar{\*x}_{ub}, \\
      &&& \bar{\*u}_{lb} \leq \*u \leq \bar{\*u}_{ub}, \\
      &&& \bar{\ddot{\*q}}_{lb} \leq \ddot{\*q} \leq \bar{\ddot{\*q}}_{ub}, \\
      &&& \*x(t_k) = \*x_0. \\
      \end{aligned}
\end{equation*}

\begin{equation*}
  f(x,u) = \begin{bmatrix} \dot{q} \\ M(q)^{-1} (\tau - C(q, \dot{q}) - G(q)) \end{bmatrix}
\end{equation*}

\begin{equation*}
  p(\*y | \*X, \boldsymbol{\theta}) = \dots
\end{equation*}